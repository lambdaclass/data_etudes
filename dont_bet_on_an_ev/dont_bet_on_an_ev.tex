\documentclass[12pt]{article}
\usepackage[margin=1in]{geometry} 
\usepackage{amsmath}
\usepackage{tcolorbox}
\usepackage{amssymb}
\usepackage{amsthm}
\usepackage{lastpage}
\usepackage{fancyhdr}
\usepackage{accents}
\usepackage{breqn}
\usepackage{cancel}
\pagestyle{fancy}
\usepackage{hyperref}

% above are packages, settings, and declarations

\title{Don't Bet on an Expected Value}
\author{Fernando Lopez \and Federico Carrone}

\begin{document}

\maketitle
\cfoot{\thepage\ of \pageref{LastPage}}

\subsection{To play or not to play?}

Imagine a game where you toss a fair coin. If it comes up heads your monetary wealth increases by 50\%; if it comes up tails, it is reduced by 40\%. You’re not only doing this once, but many times; for example, once per week for the rest of your life. Would you accept the rules of our game? Would you play this game if given the opportunity?

\subsection{Solution}
Every run of the game is independent and success equally likely. Thus $X_i$, a random variable returning 1 on success and 0 on failure, is nothing else than a $Bernoulli(1/2)$. $X$, the random variable that counts the number of successful outcomes in $n$ runs of the game, would be defined as:
\begin{equation*}
 X = \sum\limits_{k=1}^{n} X_i
\end{equation*}

Then $X \sim Bin(n, 1/2)$.

After $n$ coin tosses, our random variable final wealth $W_n$ can be modeled as:
\begin{align}
  \begin{equation*}
    W_n = w_0 * 1.5^X * 0.6^{n-X}
  \end{equation*}
\end{align}

In order to decide if we would accept to play this game for the rest of our lives we would have to check what happens to $W_n$ when $n \rightarrow \infty$.
\\\\

Let's add and substract $\mathbb{E}[X]$ to the exponents of $W_n$:
\begin{equation*}
  \begin{split}
    W_n &= w_0 * 1.5^{X + \mathbb{E}[X] - \mathbb{E}[X]} * 0.6^{n - X + \mathbb{E}[X] - \mathbb{E}[X]} \\
    & = w_0 *  1.5^{X - \mathbb{E}[X]}*1.5^{\mathbb{E}[X]} * 0.6^{- X + \mathbb{E}[X]} * 0.6^{n - \mathbb{E}[X]} \\
    & = w_0 *  1.5^{X - \mathbb{E}[X]} * 0.6^{- X + \mathbb{E}[X]} * 1.5^{\mathbb{E}[X]} * 0.6^{n - \mathbb{E}[X]}\\
    & = w_0 *  1.5^{X - \mathbb{E}[X]} * 0.6^{- (X - \mathbb{E}[X])} * 1.5^{\mathbb{E}[X]} * 0.6^{n - \mathbb{E}[X]}

  \end{split}
\end{equation*}

Let's define $A_n$ as:
\begin{align}
  \begin{equation*}
    A_n = 1.5^{X - \mathbb{E}[X]} * 0.6^{-(X - \mathbb{E}[X])}
  \end{equation*}
\end{align}

And $B_n$ as follows:
\begin{align}
  \begin{equation*}
    B_n = 1.5^{\mathbb{E}[X]} * 0.6^{n - \mathbb{E}[X]}
  \end{equation*}
\end{align}

Now we can express $W_n$ with $A_n$ and $B_n$:
\begin{align}
  \begin{equation*}
    W_n = w_0 * A_n * B_n
  \end{equation*}
\end{align}

Let's see what happens to $A_n$ and $B_n$ when $n \rightarrow \infty$:
\begin{equation*}
  \begin{split}
    \lim_{n\to\infty} A_n &=  1.5^{X - \mathbb{E}[X]} * 0.6^{-(X - \mathbb{E}[X])}\\
    &= \lim_{n\to\infty} 1.5^{n/2 - n/2} * 0.6^{- (n/2 - n/2} \\
    &= \lim_{n\to\infty}w_0 * 1.5^0 * 0.6^0 \\
    & = 1
  \end{split}
\end{equation*}
\begin{equation*}
  \begin{split}
    \lim_{n\to\infty} B_n &=  1.5^{\mathbb{E}[X]} * 0.6^{n - \mathbb{E}[X]} \\
    &= \lim_{n\to\infty}w_0 * 1.5^{n/2} * 0.6^{n - n/2} \\
    &= \lim_{n\to\infty}w_0 * 1.5^{n/2} * 0.6^{n/2} \\
    &= \lim_{n\to\infty}w_0 * (\sqrt{1.5*0.6})^{n}\\
    &= \lim_{n\to\infty}w_0 * (\sqrt{0.9}) ^{n}\\
    &= 0
  \end{split}
\end{equation*}

Now let's see what happens to $W_n$ when $n \rightarrow \infty$:
\begin{equation*}
  \begin{split}
    \lim_{n\to\infty} W_n &= w_0 * A_n * B_n\\
    & = w_0 * 1 * 0 \\
    & = 0
  \end{split}
\end{equation*}


  
Our wealth will decrease to 0 when $n\to\infty$ independently of our starting wealth. The answer to our initial question should be: no, I do not want to play since I'm certain to go bust.

\subsection{The expected value}
A common erroneous way of approaching the problem is to calculate the expected value of your wealth:
\begin{equation*}
  \begin{split}
    \mathbb{E}[W_n] &= \mathbb{E}[w_0 * 1.5^X * 0.6^{n-X}]\\
    & = w_0 * 0.6^n * \mathbb{E}[(1.5/0.6)^X]\\
    & = w_0 * 0.6^n * \mathbb{E}[(2.5)^X]
  \end{split}
\end{equation*}

To calculate the expected value of $k^X$, we'll use the theorem known as the Law of the Unconscious Statistician for discrete random variables:
\begin{equation}
    \mathbb{E}[g(X)] = \sum\limits_{x \in X} g(x_i)*p_X(x_i)
\end{equation}

With the binomial pmf being:
\begin{equation}
    p_X(x) = {n \choose x} p^x (1-p)^{n-x}
\end{equation}

Then:
\begin{equation*}
  \begin{split}
    \mathbb{E}[(2.5)^X] &= \sum\limits_{x=0}^{n} 2.5^x {n \choose x} p^x (1-p)^{n-x}\\
    &= \sum\limits_{x=0}^{n} {n \choose x} (2.5p)^x (1-p)^{n-x}\\
    &= (2.5p + 1 - p)^n\\
    &= (2.5 * \frac{1}{2} + 1 - 1/2)^n\\
    & = 1.75^n
  \end{split}
\end{equation*}

Finally:
\begin{equation*}
  \begin{split}
    \mathbb{E}[W_n] &= w_0 * 0.6^n * \mathbb{E}[(2.5)^X]\\
    &= w_0 * 0.6^n * 1.75^n\\
    &= w_0 * 1.05^n
  \end{split}
\end{equation*}

This might lead us to conclude that the gamble is worth taking since we $expect$ our wealth to increase indefinitely at a rate of $1.05$ every time we flip the coin.
\\\\
Most times, expected value won't tell us if a gamble is worth taking. It tells us what would happen on average if a group of people were to take the bet on parallel, and there are some conditions that need to be satisfied to be certain that this coincides with what will happen to one individual taking the bet repeatedly over time.

\subsection{Repetition matters}
It is not the same to calculate the average return of a hundred people who go the casino one night and place just one bet than to calculate what happens to the wealth of an individual who visits the casino a hundred days in a row. The average return of a hundred people going once tells us nothing about what would happen to a person going over and over again.

\subsection{Bibliography}
\href{https://medium.com/incerto/the-logic-of-risk-taking-107bf41029d3}{The Logic of Risk Taking} - Nassim Nicholas Taleb
\\\\
\href{https://arxiv.org/abs/0902.2965}{Optimal leverage from non-ergodicity} - Ole Peters
\\\\
\href{https://arxiv.org/abs/1405.0585}{Evaluating gambles using dynamics} - Ole Peters, Murray Gell-Mann
\end{document}
