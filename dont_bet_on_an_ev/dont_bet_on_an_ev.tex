\documentclass[12pt]{article}
\usepackage[margin=1in]{geometry} 
\usepackage{amsmath}
\usepackage{tcolorbox}
\usepackage{amssymb}
\usepackage{amsthm}
\usepackage{lastpage}
\usepackage{fancyhdr}
\usepackage{accents}
\usepackage{breqn}
\usepackage{cancel}
\pagestyle{fancy}

% above are packages, settings, and declarations

\title{Don't Bet on an Expected Value}
%\author{Fernando Lopez \and Federico Carrone}

\begin{document}

\maketitle
\cfoot{\thepage\ of \pageref{LastPage}}

\section{To play or not to play?}

Imagine a game where you toss a fair coin. If it comes up heads your monetary wealth increases by 50\%; if it comes up tails, it is reduced by 40\%. You’re not only doing this once, but many times; for example, once per week for the rest of your life. Would you accept the rules of our game? Would you play this game if given the opportunity?

\section{Solution}
Every run of the game is independent and success equally likely. Thus $X_i$, a random variable returning 1 on success and 0 on failure, is nothing else than a $Bernoulli(1/2)$. $X$, the random variable that counts the number of successful outcomes in $n$ runs of the game, would be defined as:

\begin{equation*}
 X = \sum\limits_{k=1}^{n} X_i
\end{equation*}

Then $X \sim Bin(n, 1/2)$.
\\\\
After $n$ coin tosses, our random variable final wealth $W_n$ can be modeled as:
\begin{equation*}
  \begin{split}
    W_n &= g(X)\\
    W_n &= w_0 * 1.5^X * 0.6^{n-X}
  \end{split}
\end{equation*}

In order to decide if we would accept to play this game for the rest of our lives we would have to check what happens to $W_n$ when $n \rightarrow \infty$.
\\\\
According to the Strong Law of Large Numbers, which we already know holds for the Bernoulli distribution, we will prove that $X$ is equal to its expected value $np = n/2$ with probability equal to 1 when $n \rightarrow \infty$

\begin{equation*}
  \begin{split}
    \mathbb{P}(\lim_{n\to\infty} \tfrac{\sum\limits_{k=1}^{n} X_i}{n} = \mu_{X_i}) =& 1\\
    \mathbb{P}(\lim_{n\to\infty} \tfrac{\cancel{n}*\sum\limits_{k=1}^{n} X_i}{\cancel{n}} = np) =& 1\\
    \mathbb{P}(\lim_{n\to\infty} X = \tfrac{n}{2}) =& 1
  \end{split}
\end{equation*}

As $n$ grows, the possible outcomes of $W_n$ start to concentrate around an only value with probability equal to $1$. When $n \rightarrow \infty$, $W_n$ will become a degenerate random variable with no randomness. We can see that by applying $g()$ at both sides

\begin{equation*}
  \begin{split}
    \mathbb{P}(\lim_{n\to\infty} g(X) = g(\tfrac{n}{2})) = 1
    \mathbb{P}(\lim_{n\to\infty} W_n = w_0 * 1.5^{n/2} * 0.6^{n-n/2}) = 1
    \end{split}
\end{equation*}

Now, what is that only value? Lets solve the limit

\begin{equation*}
  \lim_{n\to\infty} w_0 * 1.5^{n/2} * 0.6^{n-n/2} = \lim_{n\to\infty}w_0 * 1.5^{n/2} * 0.6^{n/2} =
\end{equation*}
\begin{equation*}
    = \lim_{n\to\infty}w_0 * (\sqrt{1.5*0.6})^{n} = \lim_{n\to\infty}w_0 * (\sqrt{0.9}) ^{n} = 0
\end{equation*}
\\\\
So

\begin{equation*}
    \mathbb{P}(\lim_{n\to\infty} W_n = 0) = 1
\end{equation*}
\\\\
With probability equal to $1$, our wealth will decrease to 0 when $n\to\infty$ independently of our starting wealth. The answer to our initial question should be: no, I do not want to play since I'm certain to go bust.

\section{The expected value}
A common erroneous way of approaching the problem is to calculate the expected value of the return that modifies your wealth, a random variable that we call $V_n$:

\begin{equation*}
  \mathbb{E}[V_{n+1}] = V_{n} * \mathbb{E}[R]^n
\end{equation*}

With $R$ being a random variable defined as:
\begin{equation*}
R = \left\{
	\begin{array}{ll}
		1.5 & \mbox{with probability 1/2} \\
		0.6 & \mbox{with probability 1/2}
	\end{array}
\right.
\end{equation*}

Let's calculate $\mathbb{E}[R]$:
\begin{equation*}
  \mathbb{E}[R] = 1.5 * 0.5 + 0.6 * 0.5 = 1.05
\end{equation*}

The expected value of $R$ is 1.05. This might lead us to conclude that the gamble is worth taking since $\mathbb{E}[R] > 1$ and we $expect$ our wealth to increase.
\\\\
Expected value doesn't tell us if a gamble is worth taking. It tells us what would happen on average if a group of people were to take the bet on parallel.
\\\\
That is because it is not the same to calculate the average returns of a hundred people who go the casino one night and place just one bet than to calculate what happens to the wealth of an individual who visits the casino a hundred days in a row. The average return of a hundred people going once tells us nothing about what would happen to a person going over and over again. 

\end{document}
